\usepackage[imagemagick]{sagetex}

\usepackage{calc}
%\usepackage{svg}
\usepackage{pst-text}
\newcommand{\executeiffilenewer}[3]{%
	\ifnum\pdfstrcmp{\pdffilemoddate{#1}}%
	{\pdffilemoddate{#2}}>0%
	{\immediate\write18{#3}}\fi%
}
\newcommand{\includesvg}[1]{%
	\executeiffilenewer{#1.svg}{#1.pdf}%
	{inkscape -z -D --file=#1.svg %
		--export-pdf=#1.pdf --export-latex}%
	\input{#1.pdf_tex}%
}
\usepackage{cancel}
\usepackage{scrextend}
%\usepackage{indentfirst}
\usepackage{float}
\usepackage[brazilian]{babel}
\usepackage[latin1]{inputenc}
\usepackage{times}
\usepackage[T1]{fontenc}
\usepackage[final]{pdfpages}
\usepackage{epstopdf}
\usepackage[top=3cm,bottom=2cm,left=3cm,right=2cm,marginparsep=0pt]{geometry}
\usepackage{multirow}
\usepackage[colorlinks=true,linkcolor=blue,citecolor=red,urlcolor=black,bookmarks=true,pdfstartview=FitB]{hyperref} %uso de links
%\usepackage[hyphenbreaks]{breakurl}
%\usepackage{makeidx}
%\makeindex
\usepackage{wrapfig}
\usepackage{flashmovie}
\usepackage{fancyhdr} %cabe�alho
\usepackage{setspace} %espa�amento entre linhas

\usepackage{titlesec}    
\titleformat{\chapter}[display]
{\normalfont%
	\Large% %change this size to your needs for the first line
	\bfseries}{\chaptertitlename\ \thechapter}{20pt}{%
	\Large %change this size to your needs for the second line
}

\usepackage{xcolor}
\definecolor{verde}{rgb}{0,0.5,0}
\usepackage{showexpl} % exibe codigos de figuras
\usepackage{listings}
\lstset{
	language=C++,
	basicstyle=\ttfamily\small, 
	keywordstyle=\color{blue}, 
	stringstyle=\color{verde}, 
	commentstyle=\color{red}, 
	extendedchars=true, 
	showspaces=false, 
	showstringspaces=false, 
	numbers=left,
	numberstyle=\tiny,
	breaklines=true, 
	backgroundcolor=\color{green!10}, 
	breakautoindent=true, 
	captionpos=b,
	xleftmargin=0pt,
	emph={main, printf, scanf},
	emphstyle={\color{black}\bf},
	morekeywords={comando1, comando2},language=C++,
	basicstyle=\ttfamily\small,
	keywordstyle=\color{blue},
	stringstyle=\color{verde},
	commentstyle=\color{red},
	extendedchars=true,
	showspaces=false,
	showstringspaces=false,
	numbers=left,
	numberstyle=\tiny,
	breaklines=true,
	backgroundcolor=\color{green!10},
	breakautoindent=true,
	captionpos=b,
	xleftmargin=0pt,
}
\usepackage{amsmath}
\usepackage[font=footnotesize, labelfont=bf, margin=0.5cm]{caption} %Altera formata��o das legendas
\usepackage{indentfirst} %Espa�amento aplicado � primeira linha do primeiro par�grafo
%%%%%%%%%%%%%%%%%%%%%%
\usepackage{color}
\usepackage{ifpdf}
\ifpdf %if using pdfLaTeX in PDF mode
%\usepackage[pdftex]{graphicx}
\DeclareGraphicsExtensions{.pdf,.png,.jpg,.jpeg,.mps}
\usepackage{pgf}
\usepackage{tikz}
\else %if using LaTeX or pdfLaTeX in DVI mode
\usepackage{graphicx}
\DeclareGraphicsExtensions{.eps,.bmp}
\DeclareGraphicsRule{.emf}{bmp}{}{}% declare EMF filename extension
\DeclareGraphicsRule{.png}{bmp}{}{}% declare PNG filename extension
\usepackage{pgf}
\usepackage{tikz}
\usepackage{pstricks}
\fi
\usepackage{epic,bez123}
\usepackage{floatflt}% package for floatingfigure environment
\usepackage{wrapfig}% package for wrapfigure environment

%%%%%%%%%%%%%%%
\usepackage{cancel}
\usepackage{scrextend}
\usepackage{indentfirst}
\usepackage{float}
\usepackage[brazilian]{babel}
\usepackage[latin1]{inputenc}
\usepackage{times}
\usepackage[T1]{fontenc}
\usepackage[final]{pdfpages}
\usepackage{epstopdf}
\usepackage[top=3cm,bottom=2cm,left=3cm,right=2cm,marginparsep=0pt]{geometry}
\usepackage{multirow}
\usepackage[colorlinks=true,linkcolor=blue,citecolor=red,urlcolor=black,bookmarks=true,pdfstartview=FitB]{hyperref} %uso de links
%\usepackage[hyphenbreaks]{breakurl}
%\usepackage{makeidx}
%\makeindex
\usepackage{wrapfig}
\usepackage{flashmovie}
\usepackage{fancyhdr} %cabe�alho
\usepackage{setspace} %espa�amento entre linhas

\usepackage{titlesec}    
\titleformat{\chapter}[display]
{\normalfont%
    \Large% %change this size to your needs for the first line
    \bfseries}{\chaptertitlename\ \thechapter}{20pt}{%
    \Large %change this size to your needs for the second line
    }

\usepackage{xcolor}
\definecolor{verde}{rgb}{0,0.5,0}
\usepackage{showexpl} % exibe codigos de figuras
\usepackage{listings}
\lstset{
  language=C++,
  basicstyle=\ttfamily\small, 
  keywordstyle=\color{blue}, 
  stringstyle=\color{verde}, 
  commentstyle=\color{red}, 
  extendedchars=true, 
  showspaces=false, 
  showstringspaces=false, 
  numbers=left,
  numberstyle=\tiny,
  breaklines=true, 
  backgroundcolor=\color{yellow!20}, 
  breakautoindent=true, 
  captionpos=b,
  xleftmargin=0pt,
	emph={main, printf, scanf},
	emphstyle={\color{black}\bf},
	morekeywords={comando1, comando2},
}

\usepackage[font=footnotesize, labelfont=bf, margin=0.5cm]{caption} %Altera formata��o das legendas
\usepackage{indentfirst} %Espa�amento aplicado � primeira linha do primeiro par�grafo
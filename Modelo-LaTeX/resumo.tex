%%%%%%%%%%%%%%%%%%%%%%%%%%%%%%
\chapter*{Resumo}
\thispagestyle{empty}
\noindent
O resumo em l�ngua portuguesa dever� conter no m�nimo 150 e no m�ximo 500 palavras.
Bla Bla Bla Bla Bla Bla Bla Bla Bla Bla Bla Bla Bla Bla Bla Bla Bla Bla 
Bla Bla Bla Bla Bla Bla Bla Bla Bla Bla Bla Bla Bla Bla Bla Bla Bla Bla 
Bla Bla Bla Bla Bla Bla Bla Bla Bla Bla Bla Bla Bla Bla Bla Bla Bla Bla 
Bla Bla Bla Bla Bla Bla Bla Bla Bla Bla Bla Bla Bla Bla Bla Bla Bla Bla 
Bla Bla Bla Bla Bla Bla Bla Bla Bla Bla Bla Bla Bla Bla Bla Bla Bla Bla 
Bla Bla Bla Bla Bla Bla Bla Bla Bla Bla Bla Bla Bla Bla Bla Bla Bla Bla 
Bla Bla Bla Bla Bla Bla Bla Bla Bla Bla Bla Bla Bla Bla Bla Bla Bla Bla 
Bla Bla Bla Bla Bla Bla Bla Bla Bla Bla Bla Bla Bla Bla Bla Bla Bla Bla 
Bla Bla Bla Bla Bla Bla Bla Bla Bla Bla Bla Bla Bla Bla Bla Bla Bla Bla 
Bla Bla Bla Bla Bla Bla Bla Bla Bla Bla Bla Bla Bla Bla Bla Bla Bla Bla 
Bla Bla Bla Bla Bla Bla Bla Bla Bla Bla Bla Bla Bla Bla Bla Bla Bla Bla 
Bla Bla Bla Bla Bla Bla Bla Bla Bla Bla Bla Bla Bla Bla Bla Bla Bla Bla 
Bla Bla Bla Bla Bla Bla Bla Bla Bla Bla Bla Bla Bla Bla Bla Bla Bla Bla 
Bla Bla Bla Bla Bla Bla Bla Bla Bla Bla Bla Bla Bla Bla Bla Bla Bla Bla 
Bla Bla Bla Bla Bla Bla Bla Bla Bla Bla Bla Bla Bla Bla Bla Bla Bla Bla 
Bla Bla Bla Bla Bla Bla Bla Bla Bla Bla Bla Bla Bla Bla Bla Bla Bla Bla 
Bla Bla Bla Bla Bla Bla Bla Bla Bla Bla Bla Bla Bla Bla Bla Bla Bla Bla 
Bla Bla Bla Bla Bla Bla Bla Bla Bla Bla Bla Bla Bla Bla Bla Bla Bla Bla 

\vspace{1.0cm}
\noindent
\textbf{Palavras-chave}: palavra-chave 1. palavra-chave 2. palavra-chave 3.
% Minimo de 3 e m�ximo de 6 palavras-chave.

\listoftables
\thispagestyle{empty}
\listoffigures
\thispagestyle{empty}

\tableofcontents
\thispagestyle{empty}

\mainmatter


%%%%%%%%%%%%%%%%%%%%%%%%%%%%%%%%%%%%%%%%%%%%%%%%%%%%%%%%%%%%%%%